\section{Execution}
In order to determine the frequency response,
a negative feedback operation amplifier as shown in Figure \ref{fig:gegen} is set up,
which is adjusted via the negative feedback branch. The output voltage and the input voltage are graphically displayed on an oscilloscope.
\begin{figure}
  \centering
  \begin{circuitikz}
    \draw
    (0, 0) node[op amp] (opamp2) {}
    (opamp2.-) to[R,a=$R_1$] (-4, 0.5)
    (opamp2.-) (-1.5,0.49)--(-1.5,1.5)
    to[R,l=$R_N$] (1.19,1.5)
    to[short] (opamp2.out)
    (opamp2.+) --(-1.19,-1.5)
    ;
    \draw
    (0,-1.5)--(-4,-1.5)
    (0,-1.5)--(1.2,-1.5)
    ;
    \draw[<->]
    (-4,-1.5)--(-4,0.49);
    \draw[<->]
    (1.2,-1.5)--(opamp2.out)
    ;
    \draw[<->]
    (-1.5,0.49)--(-1.5,-1.5)
    ;
    \node [right, align=left] at (-4,-0.5){$U_1$};
    \node [left, align=left] at (-1.5,-0.5){$U_N$};
    \node [right, align=left] at (1.2,-0.7){$U_A$};

  \end{circuitikz}
  \caption{Feedback inverting linear amplifier.}
  \label{fig:gegen}
\end{figure}
To determine the terminal voltage, use the circuit shown in Figure \ref{fig:gegen} and a non-inverting electrometer amplifier shown in Figure \ref{fig:nig}.
The circuitry is then used to determine the terminal voltage.
\begin{figure}
  \centering
  \begin{circuitikz}
    \draw
    (0,0) node[op amp](oplol) {}
    ;
    \draw
    (-2,3)--(1.2,3)
    to[R,l=$R_1$] (1.2,1.5) to[R,l=$R_N$] (oplol.out)
    ;
    \draw (1.2,1.5)--(-1.2,1.5)--(oplol.-);
  \end{circuitikz}
  \caption{Non-inverting electrometer amplifier.}
  \label{fig:nig}
\end{figure}
Circuit \ref{fig:integrator} is constructed to integrate input signals using the operational amplifier.
\begin{figure}
  \centering
  \begin{circuitikz}
      \draw
      (0, 0) node[op amp] (opamp) {}
      (opamp.-) to[R,l=$R$] (-3, 0.5)
      (opamp.-) |- (-1, 2) to[C,a=$C$] (1, 2) -| (opamp.out)
      ;
  \end{circuitikz}
  \caption{Reverse integrator.}
  \label{fig:integrator}
\end{figure}
In order to differentiate an input signal, the resistor and capacitor in Figure \ref{fig:integrator}
are swapped according to Figure \ref{fig:diff}.
\begin{figure}
  \centering
  \begin{circuitikz}
      \draw
      (0, 0) node[op amp] (opamp3) {}
      (opamp3.-) to[C,l=$C$] (-3, 0.5)
      (opamp3.-) |- (-1, 2) to[R,a=$R$] (1, 2) -| (opamp3.out)
      ;
  \end{circuitikz}
  \caption{Reverse integrator.}
  \label{fig:diff}
\end{figure}
The Schmitt trigger which works as a switch because the output voltage
changes its sign when the input voltage falls under the following condition:
\begin{equation}
  \frac{-R_1}{R_P}U_B
\end{equation}
\begin{figure}
  \centering
  \begin{circuitikz}
    \draw
    (0, 0) node[op amp] (opamp5) {}
    (opamp5.+) to[R,l=$R_1$] (-4, -0.5)
    (opamp5.+) (-1.5,-0.49)--(-1.5,-1.5)
    to[R,a=$R_p$] (1.19,-1.5)
    to[short] (opamp5.out)
    ;
    \draw
    (opamp5.-)--(-4,0.49)
    ;
    \draw[<->]
    (-4,-0.49)--(-4,0.49);
    \draw[<->]
    (1.2,-1.5)--(opamp2.out)
    ;
    \draw[<->]
    (-1.5,0.49)--(-1.5,-0.49)
    ;
    \node [right, align=left] at (-4,0){$U_1$};
    \node [left, align=left] at (-1.5,0){$U_p$};
    \node [right, align=left] at (1.2,-0.7){$U_A$};

  \end{circuitikz}
  \caption{Schmitt-trigger}
  \label{}
\end{figure}
