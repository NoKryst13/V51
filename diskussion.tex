\section{Discussion}
The frequency responses measured with the linear amplifier cover the expected form of the theory. For $V'_{10}$ at low frequencies the neutral amplification
has to be very large, because the amplification of the measurement and the expected value from the resistance ratio are exactly the same. This means that the used precision
of the measurement in millivolts was not enough to get a difference between the resistance ratio and $V'$, as it is supposed in equation (\ref{eq:3}). In contrast to that for $V'_{100}$
a neutral amplification of $V = 666.67$ was approximated. This value is quite low for the neutral amplification of an operation amplifier that usually are very large values of about $10^5$ or even higher.
Therefore this is probably not a precise approximation. \\
\newline
\noindent
The product of bandwidth and $V'$ is supposed to be constant, but had differences between the two measurements, respectively between the two fitted curves, of  $\Delta = (51 \pm 31)\%$. That is
a relatively big range, caused by the errors of the fit, with a possible minimum of $\Delta_{min} = \SI{20}{\percent}$. Looking at the figure of $V'_{100}$ it was already mentioned, that the fitted curve does not cover the
data in the inspected range that precisely. \\
To clarify that the real frequency $\nu'_g$ is probably shifted to the left the alternative calculation, where $\frac{V'}{\sqrt{2}}$ exactly cuts the direct connection between the
data, lead to $\Delta = (19 \pm 21)\%$. This is still a relatively big range but obviously closer to the condition of a constant product. \\
Further the product $\nu'_{g100} V'_{100} = (1.19+/-0.16)\cdot 10^6 Hz$ is the transit frequency, but looking at the graph it is clear that the fitted curve cuts the point where $V'_{100} = 1$ much more left.
This is one more reason to expect the actual $\nu'_g$ to be shifted to the left. \\
For the transit frequency $\nu'_{g10} V'_{10} = (7.9\pm 1.4)\cdot 10^5 Hz$ for $V'_{10}$ in contrast the calculated value and the graph cover each other nicely.
Concluding this it is verified, that the operational amplifier works more stable at low amplifications.\\
\newline
\noindent
The different measurements of the terminal voltage verify as expected, that the linear amplifier is not suited for measuring high-resistance voltage sources, as the output signal is easily influenced by changes
of the input resistance. However the electrometer amplifier showed no difference, whether there was an additional input resistance or not, and therefore is suited much better for these kinds of measurement. \\
\newline
\noindent
The conditions, $U_A \sim \frac{1}{\omega}$ for the inverting integrator and $U_A \sim \omega$ for the inverting differentiator, were both verified. For the inverting integrator the operating range could be limited
to about $<\SI{3}{\kilo\hertz}$ but for the inverting differentiator no such limit was found, probably because not enough frequency range was covered in the measurement. \\
The thermal prints of both show very nicely how the circuit is able to integrate/differentiate (and invert) different input signals.\\
\newline
\noindent
For the last part with the Schmitt-Trigger, the trigger function, adjusted by the resistance ratio, and its relation to the operating voltage $\pm U_B$ were verified
and the thermal print very nicely underlines and visualises the effect.
