\section{Execution}
For the first task the input- and output voltage in dependence of the frequency of a linear amplifier, as in figure (\ref{fig:lin}),
are measured with an oscilloscope for the two different resistance ratios $\frac{\SI{100}{\kilo\ohm}}{\SI{10}{\kilo\ohm}}$ and $\frac{\SI{20}{\kilo\ohm}}{\SI{0.2}{\kilo\ohm}}$.
The frequency is varied from about \SI{1}{\kilo\hertz} to \SI{1}{\mega\hertz}. \\
\newline
\noindent
Next, the terminal voltage of the used NF-Generator is measured firstly with the same linear amplifier used before and then with an additional resistance $R = \SI{100}{\kilo\ohm}$
switched in front of $R_1$. The same is done for an electrometer amplifier, as in figure(\ref{fig:nig}). Both resistance ratios are $\frac{\SI{20}{\kilo\ohm}}{\SI{0.2}{\kilo\ohm}}$. \\
\newline
\noindent
Then for the inverting differentiator and inverting differentiator, as in figure (\ref{fig:integrator}) and (\ref{fig:diff}), the output voltage in dependence of the frequency is measured from
\SI{50}{\hertz} to \SI{1000}{\hertz}. Additionally a sin-, triangular- and rectangular voltage are integrated/differentiated and thermal prints of input- and output voltage are taken with the oscilloscope.
It is $C = \SI{1}{\micro\farad}$ and $R = \SI{200}{\ohm}$. \\
\newline
\noindent
For the last part a Schmitt-Trigger, as in figure (\ref{fig:schmitt}), is build up and the trigger voltage for a resistance ratio of $\frac{\SI{10}{\kilo\ohm}}{\SI{100}{\kilo\ohm}}$ is measured, as well
as the value $2 U_B$. Adding to this a thermal print is taken. \\
\newline
\noindent
Furthermore it is important to keep the input voltage as low as possible for the measurements, because the nearer the OA
works at his voltage limits the lesser the precision of the measurement will be.
